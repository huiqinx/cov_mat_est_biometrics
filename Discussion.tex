\section{\label{sec:discussion}Discussion}
\label{discussion}

The class of separable covariance matrix estimators \eqref{separable} that we proposed in this paper appears to be very promising. Many existing procedures already explicitly or implicitly target this class, and our proposed estimate \eqref{proposed} of the optimal separable estimator outperforms a number of existing covariance matrix estimators. This is surprising because our approach vectorizes the matrix and therefore cannot take matrix structure, such as positive-definiteness, into account. This suggests that a vectorized approach combined with a positive-definiteness constraint may have improved performance. The resulting estimator would necessarily not be separable, because the estimate of the $jk$th entry would depend on more than just the $j$th and $k$th observed features, so the $g$-modeling estimation strategy is insufficient. More work is needed.

Though our estimator performs well in simulations and in real data, providing theoretical guarantees is difficult. In the standard mean vector estimation problem with $Y_i \sim N(\theta_i, 1)$, \citet{jiang2009general} showed that an empirical Bayes estimator based on a nonparametric maximum likelihood estimate of the prior on the $\theta_i$ can indeed asymptotically achieve the same risk as the oracle optimal separable estimator. However, this was in a simple model with a univariate prior distribution. \citet{saha2020nonparametric} extended these results to multivariate $\bs{Y}_i \sim N(\bs{\theta}_i, \bs{\Sigma}_i)$ with a multivariate prior on the $\bs{\theta}_i$, but assumed that the $\bs{Y}_i$ were independent. In contrast, our covariance matrix estimator is built from arbitrarily dependent $\bs{A}_{jk}$. These imposes significant theoretical difficulties that will require substantial work to address; we leave this for future research.

Finally, we have so far assumed that our data multivariate normal. To extend our procedure to non-normal data belonging to a parametric family, we can simply modify the density function $f(\cdot \mid \bs{\eta})$ in the nonparametric maximum compositive likelihood problem \eqref{Gp hat} and in our proposed estimator \eqref{proposed}. If $f$ is unknown or difficult to specify, alternative procedures may be necessary to approximate the optimal separable rule.
